\section{Definitions and Notation}




We model a hardware module as a tuple $\hardwaredesign = (\concstates,
\concexecution, \concinputs)$, where
\begin{itemize}
\item $\concstates: \{(\variable_0,\concvalue), (\variable_1,\concvalue), \ldots, (\variable_n,\concvalue)\}$ is the finite set of state variables and their valuation,
\item $\concexecution: \{\concstates\} \times \{\concinputs\} \rightarrow 
  \{\concstates\}$ is the transition function of the module, and
\item $\concinputs: \{(\inputvariable_0,\concvalue), (\inputvariable_1,\concvalue),
  \ldots, (\inputvariable_m,\concvalue)\}$ is the set of input parameters and their valuation.
\end{itemize}


We model a symbolic execution engine as a transducer $\symbolicexecutionengine =
(\symstates, \rootsymstate, \symexecution, \pathconditions, \symalphabet,
\trees)$, where
\begin{itemize}
\item $\symstates: \{(\variable_0,\symvalue), (\variable_1,\symvalue), \ldots, (\variable_n,\symvalue)\}$
\item $\rootsymstate \in \symstates$
\item $\symexecution: \{\symstates,\pathcondition\}
  \times \{\syminputs\} \rightarrow \trees$
\item $\pathconditions$
\item $\symalphabet$ The alphabet of symbols that appear in symbolic expressions
  and symbolic assignments.
\item $\trees$ The set of trees $\tree = \{E,\nodes\}$. Each tree is a binary tree of nodes.
\end{itemize}


\begin{itemize}
\item $\pathcondition \in \pathconditions$ is the path condition of a particular node of a tree.
\item $\pathconditions \subseteq \mathrm{symexpressions}$ Path condition is a subtype of
  symexpressions. The set of all path conditions is a subset of the set of all
  symexpressions.
\item $\nodes: \{\symstate,\pathcondition\}$, where $\symstate \in \symstates$
  and $\pathcondition \in \pathconditions$.
\item $\syminputs: <(\inputvariable_0, \symexpression), (\inputvariable_1,
  \symexpression), \ldots, (\inputvariable_m, \symexpression)>$
\end{itemize}

A symbolic expression \symexpression{} is an expression involving at least one
symbol in \symalphabet. The expression may contain zero or more concrete
literals, arithmetic operators, and logical operators. Examples of
symbolic expressions include `$\alpha$' and `$\alpha + 1 \le \beta$,' where $\alpha,
\beta \in \symalphabet$.


%% We model the program to be symbolically executed
%% S = {(reg_0, symexpr_0), (reg_1, symexpr_1), ..., (reg_n, symexpr_n)} (edited)
%% symInp = {(inp_0, symexpr_0), (inp_1, symexpr_1), ..., (inp_k, symexpr_k)} (edited)
%% Inp = {(inp_0, inpval_0), (inp_1, inpval_1), ..., (inp_k, inpval_k)} (edited)
%% C = {(reg_0, concval_0), (reg_1, concval_1), ..., (reg_n, concval_n)}

\subsection{Properties of Symbolic Execution}
King formalized the use of symbolic execution~\cite{king1976symbolic} and describes three
properties provided by symbolic execution. We name and summarize the properties
here.
\setcounter{property}{0}
\renewcommand{\theproperty}{K.\arabic{property}}
\begin{property}[Sound Paths]
  \label{prop:kingsound}
  The path condition $\pathcondition$ never becomes unsatisfiable. This means that for each
  leaf node the path condition $\pathcondition$ associated with that leaf node has at
  least one concrete valuation which would drive execution down the same path of
  execution.
  
  We express this the following way: 
  
  \begin{align*}
  \forall a = \mathtt{maptoalph}(\mathrm{alphabet}), & n =
\mathtt{intree}(\symexecution(\symstate,\syminput)), \\
 & \mathtt{simplify}(\mathtt{plugin}(n.\pathcondition,a) = \texttrue.
  \end{align*}
  
\end{property}


\begin{property}[Unique Paths]
  \label{prop:kingunique}
The path condition $\pathcondition_1$ and $\pathcondition_2$ associated with any two paths of the
tree are mutually unsatisfiable. In other words, there exists no concrete
valuation that could drive execution down two distinct paths of the symbolic
execution tree.

We express this the following way: 

\begin{align*}
\forall a = \mathtt{maptoalph}(\mathrm{alphabet}), &  n_1 n_2 =
\mathtt{intree}(\symexecution(\symstate,\syminput)) \\
& \wedge
n_1 \neq n_2 \\
&  \wedge
(\mathtt{ischildof}(n_1, n_2) \vee  \mathtt{ischildof}(n_2, n_1))= \textfalse \\
&\rightarrow \mathtt{simplify}(\mathtt{plugin}(l.\pathcondition,a) = \textfalse
\end{align*}
\end{property}

\begin{property}[Commutativity]

  We define some functions.
  \begin{itemize}
  \item maptoalph: $\{\symalphabet\} \rightarrow \{\concvalue\}$
  \item plugin: $\{\symexpression\} \times \{\concvalue\} \rightarrow
    \{\concexpression\}$
  \item simplify: $\{\concexpression\} \rightarrow \{\concvalue\}$
  \end{itemize}

The instantiation of some symbolic state $l \in \symstate$ can be defined as: 
$\mathtt{instantiate}(l) = \mathtt{simplify} (\mathtt{plugin}(l.\pathcondition,  \mathtt{maptoalph}(\mathrm{alphabet}))) $.


\begin{align*}
\forall a = \mathtt{maptoalph}(\mathrm{alphabet}), & l =
\mathtt{isleaf}(\symexecution(\symstate,\syminput)) \wedge
\mathtt{simplify}(\mathtt{plugin}(l.\pathcondition,a) = \texttrue \\
&\rightarrow \mathtt{simplify}(\mathtt{plugin}(l.\symstate,a)) = \\
&\qquad\concexecution(\mathtt{simplify}(\mathtt{plugin}(\symstate)),\mathtt{simplify}(\mathtt{plugin}(\syminput)))
\end{align*}


\end{property}



