\documentclass[a4paper]{article}

%% Language and font encodings
%\usepackage[english]{babel}
%\usepackage[utf8x]{inputenc}
%usepackage[T1]{fontenc}

%% Sets page size and margins
\usepackage[a4paper,top=3cm,bottom=2cm,left=3cm,right=3cm,marginparwidth=1.75cm]{geometry}

%% Useful packages
\usepackage{amsmath}
\usepackage{graphicx}
\usepackage[colorinlistoftodos]{todonotes}
\usepackage[colorlinks=true, allcolors=blue]{hyperref}

\newtheorem{define}{Definition}
\newtheorem{thm}{Theorem}
\newtheorem{property}{Property}
\newtheorem{lemma}{Lemma}
\newtheorem{cor}{Corollary}

\newcommand{\cks}[1]{\textcolor{red}{cks: #1}}
\newcommand{\ab}[1]{\textcolor{cyan}{ab: #1}}

\title{A Recursive Strategy for Symbolic Execution Expressed in Coq}
\author{A. Byrnes and C. Sturton}

\begin{document}
\maketitle

\begin{abstract}
Prior work has proposed the use of symbolic execution for the security
validation of a processor design. The approach uses a three-part recursive search strategy
that is designed to counter the exponential path growth inherent in multi-cycle
processor execution. In this work, we examine the search strategy in order to prove its correctness. We formulate the
strategy in Coq, and we find that the
strategy as proposed is not guaranteed to produce correct results---that is, the
search may find a symbolic path for which no concrete path from the processor's initial state
to the given error state exists. We tighten the requirements of two of the three
requirements of the strategy and then prove the modified strategy correct. 
\end{abstract}

\section{Introduction}
%\cks{15 pages, excl. bib, excl. appdx}

Researchers have recently begun exploring the use of software-style symbolic execution for the
verification of hardware designs~\cite{mukherjee2015hardware,liu2009star}. Symbolic execution has a proven track record in the software community as a
bug-finding tool~\cite{cadar08,cha12,do16} and as an aid in formal
verification~\cite{chi16,davidson13}. However, bringing these benefits to bear on hardware
designs has been a challenge---the complexity of the search space of relatively
simple hardware designs more closely resembles that of large, continuously interactive
software systems than that of the stand-alone software programs that are the classic
targets of symbolic execution. In response to this challenge a recent paper proposed a recursive
search strategy for symbolic execution~\cite{zhang2018recursive}. Zhang et
al.~showed the strategy to be
practical~\cite{zhang2018end}, but the soundness of the approach has not been
demonstrated. A proof of correctness is needed.

In this paper we prove that a recursive search strategy for the
symbolic execution of hardware designs is sound: a list of
constraints returned by a successful search defines a set of concrete input
sequences, each of which will take the processor from its initial reset state to
an error state.

Our goal is to validate the search strategy itself rather than any one implementation
of it. Therefore we decouple our model of symbolic execution from the particular
programming language to be symbolically executed. We formalize an abstract model of symbolic
execution in Gallina, the specification language of the Coq proof
assistant~\cite{coqreference}; the structure of the model is built
around the three fundamental properties of symbolic execution as first laid out
by King in 1976~\cite{king1976symbolic}. This has the advantage of providing soundness
guarantees for the recursive search strategy when implemented by any symbolic
execution engine which abides by the three King properties.

The symbolic exploration of a program produces a rooted, binary tree. The root
node represents the entry point of the program, and each path from root to a
leaf node represents one possible path of the execution through the program.
Although the mechanics are similar, the symbolic exploration of a hardware
design differs from this model conceptually: The tree produced by the symbolic exploration of a
hardware design represents the state transitions possible in a single clock
cycle from a given state (the root node). A sequence of state transitions, for
example from the hardware's initial, reset state to an error state is
represented by a sequence of trees. The set of all possible $n$ transitions from
a given state requires a tree (of depth $n$) of trees.

The strategy proposed by Zhang and Sturton tackles this search complexity with a
recursive search. First the $n^{\mathrm{th}}$ tree in the desired sequence of
trees is found, then the $n-1^{\mathrm{st}}$ tree, and so on, until the first
tree in the sequence, whose root node starts at the reset state, is
found.\footnote{The problem is undecidable in the general case, and Zhang et
  al. introduce a set of heuristics to make it work in
  practice~\cite{zhang2018end}.} The strategy lays out three properties that, if
satisfied at each iteration of the search, are sufficient to ensure the final
sequence of paths through trees represents a possible, multi-cycle sequence of
state transitions from the hardware's initial state to the desired (error)
state.

The bulk of our proof is a proof by induction over the sequence of trees to show
that the trees are correctly stitched together. In the base case we show that
symbolic
execution, as defined by King, implies the correct \emph{concretization} of a path through a
single symbolic execution tree as defined by Zhang and Sturton. In the inductive
step we show that a path through a sequence of trees stitched together according
to the Zhang and Sturton requirements will yield a correct concretization of a
path through the sequence of trees. Finally, we prove that the sequence begins in a reset state and ends in the
desired error state.

We find that the recursive search strategy as originally proposed is not sound---that is, the
search may find a symbolic path for which no concrete path from the processor's initial state
to the given error state exists. We tighten one of the three
requirements of the strategy and then prove the modified strategy is sound.

We present two contributions of this work:
\begin{itemize}
  \item An abstract model of symbolic execution expressed in the Coq
    framework. This model can form the starting point for 
    building a provably correct implementation of a symbolic execution engine for a
    particular language. We make our model available online.
\item We verify the soundness of the recursive search strategy for symbolic
  execution. We find and fix a flaw in the original formulation of the
  strategy. The recursive search strategy was originally developed for
  the verification of hardware designs; however, any symbolic execution engine that
  implements the proven strategy will be assured the same guarantee of soundness.
\end{itemize}

\section{Prior Work}
Symbolic execution was first formally defined by King in 1976~\cite{king1976symbolic} and has
been widely adopted by the software engineering and software security
communities. (See Schwartz et al.~\cite{schwartz2010all} for a recent survey.)

Symbolic execution has since shown practical value through implementations. Anand et al. introduced a tool JPF-SE that works as an extension of the Java PathFinder model checker to allow for symbolic execution of Java programs~\cite{anand2007jpf}.  P{\u{a}}s{\u{a}}reanu et al. later introduced the tool Symbolic PathFinder that also utilizes Java Path Finder to symbolically execute Java bytecode~\cite{puasuareanu2010symbolic}. Noller et al. recently released a tool that utilizes Java PathFinder to perform ``shadow symbolic execution'' on Java bytecode~\cite{Noller2018}.

\subsection{Use of Symbolic Execution in Hardware}
In addition to the paper by Zhang and Sturton that lays out the three-part
strategy we prove here~\cite{zhang2018recursive}, there are a handful of papers that examine
the use of symbolic execution in hardware. In a subsequent paper, Zhang et
al.~\cite{zhang2018recursive} make use of their recursive strategy to find new security
vulnerabilities in processor designs. Their work demonstrates that the strategy
is useful in practice.

Prior work first explored the use of software-style symbolic execution for hardware designs. The STAR tool combines symbolic and concrete simulation of hardware designs to
provide high statement and branch coverage~\cite{liu2009star}; PATH-SYMEX uses
forward symbolic execution applied to an ANSI-C model of the hardware
design~\cite{mukherjee2015hardware}. Both tools use a forward search strategy
and have limits on how deep or broadly they can search.

\subsection{Formalization of Symbolic Execution}

Some work has been done to formalize symbolic execution. 
Arusoaie et al. provide a formal framework for a symbolic execution tool that is language-independent~\cite{arusoaie2014generic,arusoaie2015symbolic,lucanu2017generic}. 
However, they don't provide a formal representation of King's branching properties~\cite{king1976symbolic} nor do they formally verify their tool or its outputs. 

\subsection{Backward Search Strategies in Symbolic Execution}

A backward search strategy for symbolic execution of software has been
studied~\cite{ma2011directed,chandra09,dinges04,charreteur10}. Although, to
  the best of our knowledge, the
  requirements for soundness have not been examined in the literature. 
  There the approach is to search
backward through function call chains from a goal line of code to the program's
entry point. The strategy is hindered by complications that arise in software,
such as floating point calculations and external method calls. To the best of our knowledge, the strategy has not been formalized
in the software community, nor its validity proven.


\section{System Framework}
\section{Verification Approach}
\section{Conclusion and Future Work}

\input scraptext
%% \input introduction
%% \input background
%% \input proofdesign
%% \input relatedwork
%% \input conclusion


\bibliographystyle{alpha}
\bibliography{references}

\end{document}
