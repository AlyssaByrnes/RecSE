\documentclass[a4paper]{article}

%% Language and font encodings
%\usepackage[english]{babel}
%\usepackage[utf8x]{inputenc}
%usepackage[T1]{fontenc}

%% Sets page size and margins
\usepackage[a4paper,top=3cm,bottom=2cm,left=3cm,right=3cm,marginparwidth=1.75cm]{geometry}

%% Useful packages
\usepackage{amsmath}
\usepackage{graphicx}
\usepackage[colorinlistoftodos]{todonotes}
\usepackage[colorlinks=true, allcolors=blue]{hyperref}

\newtheorem{define}{Definition}
\newtheorem{thm}{Theorem}
\newtheorem{property}{Property}
\newtheorem{lemma}{Lemma}
\newtheorem{cor}{Corollary}

\title{A Recursive Strategy for Symbolic Execution Expressed in Coq}
\author{A. Byrnes and C. Sturton}

\begin{document}
\maketitle

\begin{abstract}
Prior work has proposed the use of symbolic execution for the security
validation of a processor design. The approach uses a three-part recursive search strategy
that is designed to counter the exponential path growth inherent in multi-cycle
processor execution. In this work, we examine the search strategy in order to prove its correctness. We formulate the
strategy in Coq, and we find that the
strategy as proposed is not guaranteed to produce correct results---that is, the
search may find a symbolic path for which no concrete path from the processor's initial state
to the given error state exists. We tighten the requirements of two of the three
requirements of the strategy and then prove the modified strategy correct. 
\end{abstract}

\section{Introduction}
\section{Prior Work}
Symbolic execution was first formally defined by King in 1976~\cite{king1976symbolic} and has
been widely adopted by the software engineering and software security
communities. (See Schwartz et al.~\cite{schwartz2010all} for a recent survey.)

Symbolic execution has since shown practical value through implementations. Anand et al. introduced a tool JPF-SE that works as an extension of the Java PathFinder model checker to allow for symbolic execution of Java programs~\cite{anand2007jpf}.  P{\u{a}}s{\u{a}}reanu et al. later introduced the tool Symbolic PathFinder that also utilizes Java Path Finder to symbolically execute Java bytecode~\cite{puasuareanu2010symbolic}. Noller et al. recently released a tool that utilizes Java PathFinder to perform ``shadow symbolic execution'' on Java bytecode~\cite{Noller2018}.

\subsection{Use of Symbolic Execution in Hardware}
In addition to the paper by Zhang and Sturton that lays out the three-part
strategy we prove here~\cite{zhang2018recursive}, there are a handful of papers that examine
the use of symbolic execution in hardware. In a subsequent paper, Zhang et
al.~\cite{zhang2018recursive} make use of their recursive strategy to find new security
vulnerabilities in processor designs. Their work demonstrates that the strategy
is useful in practice.

Prior work first explored the use of software-style symbolic execution for hardware designs. The STAR tool combines symbolic and concrete simulation of hardware designs to
provide high statement and branch coverage~\cite{liu2009star}; PATH-SYMEX uses
forward symbolic execution applied to an ANSI-C model of the hardware
design~\cite{mukherjee2015hardware}. Both tools use a forward search strategy
and have limits on how deep or broadly they can search.

\subsection{Formalization of Symbolic Execution}

Some work has been done to formalize symbolic execution. 
Arusoaie et al. provide a formal framework for a symbolic execution tool that is language-independent~\cite{arusoaie2014generic,arusoaie2015symbolic,lucanu2017generic}. 
However, they don't provide a formal representation of King's branching properties~\cite{king1976symbolic} nor do they formally verify their tool or its outputs. 

\subsection{Backward Search Strategies in Symbolic Execution}

A backward search strategy for symbolic execution of software has been
studied~\cite{ma2011directed,chandra09,dinges04,charreteur10}. Although, to
  the best of our knowledge, the
  requirements for soundness have not been examined in the literature. 
  There the approach is to search
backward through function call chains from a goal line of code to the program's
entry point. The strategy is hindered by complications that arise in software,
such as floating point calculations and external method calls. To the best of our knowledge, the strategy has not been formalized
in the software community, nor its validity proven.


\section{System Framework}
\section{Verification Approach}
\section{Conclusion and Future Work}

\input scraptext
%% \input introduction
%% \input background
%% \input proofdesign
%% \input relatedwork
%% \input conclusion


\bibliographystyle{alpha}
\bibliography{references}

\end{document}
