\section{Background}
\cks{Short primer on SE; statement of King properties; explanation of RecSE; statement of three RecSE properties}
Symbolic execution is a means of exploring many paths through a program
in order to find paths leading to an error state or bug. In symbolic execution
input variables to a program are given symbolic values. As the program executes
the symbolic values are used in place of the usual concrete literals and the
resulting symbolic expressions may propagate throughout the program's
state. Figure~\ref{fig:se} demonstrates the idea. For example, for the code in
Figure~\ref{fig:sea}, if $\mathtt{reset}$ and $\mathtt{count}$ are initialized
with the symbolic values $r_0$ and $c_0$, respectively, then after symbolically
executing lines 1, 3, and 4 count may be set to the symbolic expression $c_0 +
1$ as shown in Figure~\ref{fig:sec}. In addition to the (partially) symbolic
state that is maintained, a symbolic execution engine keeps track of the
\emph{path condition}. The path condition is a conjunction of propositions that
accrue at each conditional branch point in the program. There is one path
condition per path of execution through the code. In Figure~\ref{fig:sec} the
path condition for the path through lines 1, 3, 4, 5, 6 is shown. When execution
reaches the $\mathtt{ERROR}$ at line 6 the path condition is $\mathit{pc} := r_0
== 0 \wedge c_0 + 1 > 3$. This expression can then be solved using a standard,
off-the-shelf SMT solver to find a satisfying solution, say $r_0 := 0$ and $c_0
:= 3$. Substituting these values for $\mathtt{reset}$ and $\mathtt{count}$,
respectively and executing the code concretely could cause execution to follow
the same path as was followed symbolically.

More formally, ...
