\section{Recursive Search Strategy}
The goal of the symbolic execution
of a hardware design is to find a sequence of $n$ inputs that will take the hardware
module from an initial reset state to an error state in $n$ clock cycles.
 It starts with a node $s_i$ that contains an error state, and uses symbolic execution to find 
 a state $s_{i-1}$ that has $s_i$ as one of its possible next-state transitions.  
This search continues until one of the found $s$'s is a reset state. 
Each application of symbolic execution produces a tree with a particular leaf node of interest---the desired next-state $s_i$.
The path conditions of each of these leaf nodes can be used to find an input leading to the next leaf node, 
producing a string of inputs leading to an error state.


We express the three original requirements of the recursive strategy as
laid out by Zhang et al.~\cite{zhang2018recursive}. 
%The following requirements, when placed on our system, should be enough to prove that the strategy works as expected:
We start by formally defining two instantiation operations that are only informally described
by Zhang et al. but that are used by the strategy requirements.

\subsection{Instantiation Operations} 

The root node of a symbolic execution tree \tree{} contains a symbolic state
  ($\mathtt{getroot}(\tree).\vv{\symstate}$), which can be viewed as a representation of
  a set of possible concrete states. If $\vv{\symstate}$ is fully symbolic, meaning
  that all registers in the design are assigned symbolic expressions ($[(r_0,e_0),(r_1,e_1),\ldots]$), then it represents the set of all
  possible concrete states $\concstates$ (including states that may be
  unreachable from the initial state). From the set of all possible concrete
  states, all possible next states are reachable, and these are represented by
  the leaves of the symbolic execution tree. The symbolic state of each leaf node of the tree
  represents a subset of $\concstates$.

  There are two instantiation operations: $\mathit{concretizeRoot}$ and
  $\mathit{concretizeLeaf}$. The effect of the first is to find a particular
  subset of the concrete states represented by a root node of a symbolic
  execution tree. The effect of the second is to find the set of concrete states
  represented by a particular leaf node of a symbolic execution tree. 
%% Our two instantiation operations are defined in the following way: 
%% $$\mathtt{concretize\_root} : \{ \tree \} \rightarrow \{\concstates\}$$

%% $$\mathtt{concretize\_leaf} : \{ \tree \} \rightarrow \{\concstates\}$$

%% They are bound by the following requirements:

\begin{mydefinition}[$\mathtt{concretizeRoot}$]
  \label{def:concroot}
  
  For any given leaf node of the
  tree, the set of concrete states represented by that node is reachable, with a
  given input, from a subset of the concrete states represented by the root
  node. The $\mathtt{concretizeRoot}$ instantiation operation finds that subset of concrete states
  for the particular leaf node of interest of a tree. The operation takes as
  input a tree \tree{} and a particular leaf node of interest $l$ and returns
  a set of concrete states. We define it formally here.

  \begin{align*}
\forall~ \tree \in \trees,~n \in \nodes,~ \concstate \in \concstates,~& \concstate \in
\mathtt{concretizeRoot}(\tree,n) \leftrightarrow \\
&\exists~ r,l \in \tree,~m \in \textsf{SymbolicMap} \mathrm{~such~that}\\
&r = \mathtt{getroot}(\tree),~ l = \mathtt{isleaf}(\tree),~ l = n,\\
&\mathtt{evaluatePC}(l.\pathcondition, m) = \texttrue~ \mathrm{and} \\
&c = \mathtt{instantiateState}(r.\symstate, m).
    \end{align*}
\end{mydefinition}

%%   \begin{align*}
%% \forall \tree \in \trees,~ \concstate \in \concstates,~& \concstate \in
%% \mathit{concretizeRoot}(t) \leftrightarrow \\
%% &\exists r,n \in t,~ m = \mathtt{map(\cdot)} \mathrm{~such~that}\\
%% &r = \mathtt{getroot}(t), n = t.l,
%% \mathtt{simplify}(\mathtt{applymap} (n.\pathcondition, m)) = \texttrue \\
%% &\mathrm{and~}  c = \mathtt{simplify}(\mathtt{applymap}(r.\symstate, m)),
%%     \end{align*}
%% \end{mydefinition}


Where \trees{} is the set of all possible symbolic execution trees, \nodes{} is
the set of all possible tree nodes, and
\concstates{} is the set of all possible concrete states.

The definition says that a concrete state \concstate{} is in the set returned by
$\mathtt{concretizeRoot}$ if and only if there exists a mapping from symbols to
literals that makes the path condition of the leaf node evaluate to true
and, when applied to the symbolic state of the root node, produces
\concstate{}. 

\begin{mydefinition}[$\mathtt{concretizeLeaf}$]
  \label{def:concleaf}

  This instantiation operation finds the set of concrete states represented by
  the particular leaf node of interest of a tree.
  \begin{align*}
\forall ~\tree \in \trees,~n \in \nodes,~\concstate \in \concstates,~& \concstate \in
\mathtt{concretizeLeaf}(\tree,n) \leftrightarrow \\
&\exists~ l \in \tree,~ m \in \textsf{SymbolicMap} \mathrm{~such~that}\\
&l = \mathtt{isleaf}(\tree), l = n,\\
&\mathtt{evaluatePC}(l.\pathcondition, m) = \texttrue~\mathrm{and}  \\
&c = \mathtt{instantiateState}(l.\symstate, m).
    \end{align*}
\end{mydefinition}


%%   \begin{align*}
%% \forall \tree \in \trees,~ \concstate \in \concstates,~& \concstate \in
%% \mathtt{concretize\_leaf}(t) \leftrightarrow \\
%% &\exists n \in t,~ m = \mathtt{map(\cdot)} \mathrm{~such~that}\\
%% &n = t.l,
%% \mathtt{simplify}(\mathtt{applymap} (n.\pathcondition, m)) = \texttrue \\
%% &\mathrm{and~}  c = \mathtt{simplify}(\mathtt{applymap}(n.\symstate, m)),
%%     \end{align*}
%% \end{mydefinition}

Where \trees{} is the set of all possible symbolic execution trees, \nodes{} is
the set of all possible tree nodes, and
\concstates{} is the set of all possible concrete states.


The definition says that a concrete state \concstate{} is in the set returned by
$\mathtt{concretizeLeaf}$ if and only if there exists a mapping from symbols to
literals that makes the path condition of the leaf node evaluate to true and,
when applied to the symbolic state of the leaf node, produces \concstate{}.

Next, we express and prove the following lemma, stating that for root $r$ and selected leaf $l$ in tree $\tree \in \trees$ , the execution of $\mathtt{concretizeRoot}(r,l)$ for all valid inputs will result in a state $\in \mathtt{concretizeLeaf}(\tree)$
\begin{lemma} \label{cop}
$\forall s \in \textsf{SymbolicAssignment}, i \in \textsf{SymbolicInputAssignment}, c \in \textsf{Assignment}, and m \in \textsf{SymbolicMapping},$
if $c \in \mathtt{concretize\_root}(t),$
then 
$conc\_ex(c, \mathtt{get\_input} (i, m)) \in \mathtt{concretize\_leaf}(\tree),$
where $l$ is a leaf of $\tree =  \symexecution(s, i)$.
\end{lemma} 

In order to prove this, we utilize the commutativity property expressed earlier.

\subsection{Three Original Requirements} This sequence of trees ($\mathit{tree\_list}$)
must satisfy these three requirements as laid out by Zhang et al.
\setcounter{property}{0}
\renewcommand{\theproperty}{Z.\arabic{property}}
\begin{property}[Start in initial state]
  \label{prop:startinit} The leaf node of interest in the first tree in the
  sequence must be reachable from the initial, reset state.
  \begin{align*}
    \concstate_0 \in \mathtt{concretize\_root}(\mathit{tree\_list}[0]).
  \end{align*}
\end{property}

\begin{property}[End in error state]
  \label{prop:enderror} The leaf node of interest in the last tree in the
  sequence must include, in the set of concrete states it represents, one of the
  desired error states. 
  \begin{align*}
    \mathtt{concretize\_leaf}(\mathit{tree\_list}[n]) \cap \mathit{ER} \neq
    \emptyset,
  \end{align*}
\end{property}
where $\mathit{ER}$ represents the set of desired error states.

\begin{property}[Stitch trees together]
  \label{prop:stitch}
  Consecutive trees in the sequence must form a continuous path of
  execution. That is, the leaf node of one tree must represent a subset of the
  states represented by the root node of the subsequent tree in the sequence.
  \begin{align*}
    &\forall i, 0 \le i < n,\\
    &\quad\mathtt{concretize\_leaf}(\mathit{tree\_list}[i]) \subseteq
\mathtt{concretize\_root}(\mathit{tree\_list}[i+1]).
\end{align*}

\end{property}

\subsection{Modified Requirement}
We find the above properties are not sufficient to prove correctness of the
recursive strategy. We strengthen Property~\ref{prop:enderror} by replacing it with the following.

\setcounter{property}{1}
\renewcommand{\theproperty}{Z.\arabic{property}'} 
\begin{property}[Property~\ref{prop:enderror} correction]
  The leaf node of interest in the last tree in the
  sequence must represent a subset of the desired error states. The leaf node
  can not represent any concrete state that is not an error state.
  \label{prop:correctedz2}
  \begin{align*}
    \mathtt{concretize\_leaf}(\mathit{tree\_list}[n]) \subseteq \mathit{ER}.
  \end{align*}

 \end{property}
 
 
\subsection{Coq Representation of Recursive Strategy}
Our model of the recursive strategy in Coq requires three global variables
and two functions. The three global variables are defined as follows.

\begin{itemize}
\item $\mathit{tree\_list} : \vv{\textsf{SymExecTree}}$ is a list of symbolic
  execution trees. It represents the list returned by the recursive strategy.
\item $\mathit{init\_conc\_state} : \vv{\textsf{Assignment}}$ is a concrete
  state. It represents the initial state of the processor.
\item $\mathit{error\_states} \subseteq \{\concstate ~|~
  \vv{\textsf{Assignment}}\}$ is a set of concrete states. It represents the
  error states of the processor.
\end{itemize}

The two functions are defined as follows.
\begin{itemize}
\item $\mathtt{executeTreeList}(t : \vv{\textsf{SymExecTree}}) :
  \vv{\textsf{Assignment}}$ takes a list of symbolic execution trees and
  returns a concrete state.
\item $\mathtt{getInput}()$
\end{itemize}
  
The function $\mathtt{executeTreeList}$ uses the
    path conditions in the leaf nodes of interest of the sequence of trees to find a
    sequence of concrete input values, and then executes the model concretely to
    arrive at a final concrete state.

Additionally, we define the abstract method, $ \mathtt{get\_input} : \{\syminput\} \times \{\symalphabet\} \rightarrow \{\concstates\}$ which returns a concrete input that does not violate a given symbolic state's path constraint.
%This method is bound by the requirement,

%\begin{definition}[\emph{get\_input}]
%$ \forall$ symbolic inputs $i' \in \syminputs$, tree nodes $n$, and $m = \mathtt{map(\cdot)}$
%if $\exists$ a symbolic state $s \in \symstates$  such that
%$n$ is a leaf of the tree output by $\symexecution(s, i)$ and 
%$\mathtt{pc\_eval} (\mathtt{simplify}(\mathtt{applymap} (n.\pathcondition, m))) = \texttrue$, then
%$\mathtt{simplify}(\mathtt{applymap} (i, m)) = \mathtt{get\_input}(i)$.
%\end{definition}



