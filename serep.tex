\subsection{Formal Representation of Symbolic Execution}
In order to prove properties about symbolic execution, we created a formal, abstract representation of concrete execution and symbolic execution.
Next, we formally expressed King's property of commutativity as an axiom in order to utilize it in our tool's representation.

\subsubsection{Concrete Execution}
To represent concrete execution, we defined two abstract types: 
\begin{itemize}
\item \emph{input} represents a set of inputs into the system.
\item \emph{state\_assignments}  represents state variable assignments for the system.
\item \emph{conc\_state} represents a concrete state in the system.
\end{itemize}

\begin{define} [Concrete Execution]
$conc\_ex: \{conc\_state\} \times \{input\} \rightarrow conc\_state $.
\end{define}
To concretely execute, the \emph{conc\_ex} method takes a concrete state and an input and returns a new concrete state.


%Inductive-type Variables
%\begin{itemize}
%\item list input - list of elements of type ``input'', each element being an assignment for one input variable.
%\item list conc\_state - list of elements of type ``conc\_state'', each element being an assignment for one state variable.
%\end{itemize}

%Abstract Methods
%\begin{itemize}
%\item conc\_ex - takes a list of inputs of type ``conc\_state'' and a list of type ``input'' and outputs new list of ``conc\_state''.
%\end{itemize}


\subsubsection{Symbolic Execution}

To represent symbolic execution, we defined two abstract types:
\begin{itemize}
\item \emph{phi} is an abstract state.
\item \emph{pc} is a path constraint.
\end{itemize}

A symbolic state, or \emph{sym\_state} is a tuple of one item of type \emph{phi} and one item of type \emph{pc}.
A symbolic execution tree, or \emph{SE\_tree} is a tree structure consisting of \emph{sym\_state} elements. 


Additionally, we defined the following abstract methods:
\begin{define} [Symbolic Execution]
$sym\_ex: sym\_state \rightarrow SE\_tree $.
\end{define}

\begin{define} [PC Evaluation]
$pc\_eval: \{sym\_state\} \times \{input\} \times \{state\_assignments\} \rightarrow \{True, False\} $.
\end{define}
\emph{pc\_eval} takes the pc of a sym\_state, an input and conc\_state assignments, and returns $True$ if those assignments do not violate the path constraint.

\begin{define} [Symbolic State Instantiation]
$sym\_instantiate: \{phi(sym\_state)\} \times \{input\} \times \{state\_assignments\} \rightarrow conc\_states $.
\end{define}
 \emph{instantiate} takes the phi of a sym\_state, n input and conc\_state assignments, and instantiates them into a new list of conc\_states.
 



\subsubsection{Commutativity}
In order to assume the correctness of symbolic execution, we need to assume the following commutativity property as expressed by King (citation):
``The operation of instantiating the symbols
{$\alpha_i$} with specific integers, say {$j_i$}, and the operation of
executing the program are interchangeable. That is, if
one conventionally executes a program with a specific
$\alpha_i$'s by
$j_i$'s first, followed by execution), the results will be the
same as executing the program symbolically and then
instantiating the symbolic results (assigning $j$'s to $\alpha$'s).
The meaning of ``instantiating the symbolic results" is
first, for each terminal leaf in the execution tree, substitute
$j$'s for $\alpha$'s in all program variable values and in
pc. Then the ``results" are the values for the terminal 
node whose pc becomes true.'' (citation)


Our interpretation of this property was the following axiom:
\begin{axiom}
for all input lists $li$, conc\_state lists $lcs$, sym\_states $s$,
if there exists an SE\_tree $t$ such that
s is a leaf of t and
$pc\_eval (get\_pc(s), lcs, li) = true$,
then
$conc\_ex(lcs, li) = instantiate (get\_phi (x), lcs, li)$.
\end{axiom}

